\documentclass[12pt]{article}[a4paper,14pt,russian]
\usepackage[russian]{babel}
\usepackage[utf8]{inputenc}
\usepackage{color}
\definecolor{lgreen}{rgb}{0.9,1,0.8}
\definecolor{light-blue}{rgb}{0.8,0.85,1}
	\newcommand{\alarm}[1]{\textbf{ПМ "#1"}}
\begin{document}

	\section{Описание системы}
	\subsection{Аннотация}

	Система <<Программный комплекс <<ТРЕВОГА>> >> (далее ПК <<ТРЕВОГА>>) предназначена для анализа и архивирования датчиковых сообщений, полученных от абонентских терминалов, работающих в спутниковой системе <<ГОНЕЦ>>. ПК <<Тревога>> получает сообщения путем выгрузки их из системы КПУС посредством интерфейса FTP, предоставляемого системой КПУС. После выгрузки сообщений ПК <<Тревога>> производит их анализ и подготовку для представления полученной информации в удобной для пользователя форме посредством WEB браузера. ПК <<ТРЕВОГА>> осуществляет поиск специальных признаков в сообщениях, таких как нажатие тревожной кнопки, перключение питания от аккумулятора, вскрытие крышки корпуса и т.д.
	
	\subsection{Состав системы}
    ПК <<ТРЕВОГА>> состоит из следующих программных комплексов и модулей:
	\begin{enumerate}
	\item  Специальные программные модули - программные модули разработанные программистами  СС <<Гонец>>
	\item  Общесистемные программные комплексы  - программные комплексы разработанные сторонними организациями
	\end{enumerate}
    \subsubsection {Состав специальных программных модулей}
	\begin{enumerate}
	\item программный модуль отобржения данных \textbf{<<База>>} (далее ПМ <<База>>)
	\item программный модуль выгрузки и синхронизации сообщений \textbf{<<Загрузчик>>} (далее ПМ <<Загрузчик>>)
	\item программный модуль анализа данных \textbf{<<Aнализ>>} (далее ПМ <<Анализ>>)
	\end{enumerate}
    \subsubsection {Состав общесистемных программных комплексов}
    \begin{enumerate}
    \item пакет программ СУБД MySQL - свободная реляционная система управления базами данных
    \item программный модуль Samba - пакет программ, который позволяет обращаться к сетевым дискам и принтерам на различных операционных системах по протоколу SMB/CIFS. На базе этого модуля организована системах храненния сообщений, выгруженных из ПК КПУС
    \end{enumerate}
	\subsubsection{Аппаратное обеспечение системы}
	ПК <<ТРЕВОГА>> может быть развернут на одном сервере под управлением ОС Linux, но с целью повышения быстродействия системы ее можно размещать и на нескольких серверах. Одной из конфигурация является расположение системы на двух серверах:
	первый сервер используется для хранения данных, второй сервер используется для анализа данных. 
	\begin{tabular}{|c|c|}
		\hline
		Сервер & Модуль/комплекс  \\
		\hline
		Сервер 1 & ПМ <<База>> \\
		Сервер 1 & ПМ <<Загрузчик>> \\
		Сервер 1 & ПК <<Samba>> \\
		Сервер 1 & ПК СУБД <<MySQL>> \\
		Сервер 2 & ПМ <<Анализ>> \\
		
	\end{tabular}
	\section{Описание программных комплексов и программных модулей входящих в состав ПК <<Тревога>>}
	\subsection{Описание специальных программных модулей}
	\subsection{Программный модуль обработки данных ПМ <<Анализ>>}
	\subsubsection{Общее описание программного модуля}
	Программный модуль обработки данных представляет из себя приложение написанное на
	языке с++ с использование фреймворка QT версии 5.8. Программный модуль анализирует файловые сообщения, доступ к которомы получает посредством сетевой файловой системы из файлового хранилища данных. На основании анализа файловых сообщений ПМ <<Анализ>> производит заполнение базы данных, куда записывается информация полученная из файловых сообщений. ПМ <<Анализ>> осуществляет поиск специальных признаков в сообщениях и в случае их обнаружения присваивает сообщениям код <<ТРЕВОГА>>, после чего заносит данную информацию в БД
	
	\subsubsection{Программный модуль отображения данных ПМ <<База>>}

	Программный модуль отображения данных представляет из себя приложение написанное
	на языке программирования Ruby с использованием фреймворка Rails. Для своей работы требует сервер под управлением ОС Linux и установленным сервером приложений Rails, а также СУБД MySQL. Программный модуль осуществляет выборку и отображение сообщений полученных от абонентских терминалов, а также отображает координатную и датчиковую информацию содержащуюся в этих сообщениях. В случае наличия датчиковой информации, которой присвоен специальный код <<ТРЕВОГА>> , она маркируется красным цветом.
	

	
	\subsection{Программный модуль синхронизации ПМ <<Загрузчик>>}
	\subsubsection{Общее описание программного модуля}
	Программный модуль синхронизации данных ПМ <<Загрузчик>>  представляет из себя приложение, написанное на
	языке программирования Perl. Программный модуль следит за появлением новых сообщений в FTP хранилище КПУС и осуществляет выгрузку новых сообщений на сервер хранениня сообщений в рабочую и архивную папку. После успешной загрузки, сообщения удаляются с сервера КПУС. ПМ <<Загрузчик>> обеспечивает выгрузку сообщений только для одного пользовательского аккаунта КПУС. Параметры аккаунта задаются в соответствующем конфигурационном файле.
	
	\subsubsection{Минимальный состав технических средств}

	\begin{enumerate}
	\item	Для функционирования модуля необходимо иметь выделенный компьютер под управлением
	MS Windows 7 и выше либо OS Ubuntu 16
	\item Наличие сети ethernet
	\end{enumerate}
	\section {Настройка ПК <<Тревога>>}
	Настройка ПК <<Тревога>> заключается в настройке общесистемных и специальных программных модулей. В первую очеред необходимо установить общесистемные программные модули, затем специальные. 
	\subsection {Настройка общесистемных программных модулей и комплексов}
	
	\subsubsection {Настройка СУБД MySQL}
	Ключевым элементом системы ПК <<ТРЕВОГА>> является база данных под управлением СУБД MySQL. 
	
	После установки СУБД MySQL необходимо создать пользователя для доступа к базе данных и непосредственно саму базу данных для хранения информации ПК <<ТРЕВОГА>>. При создании базы данных указывается имя базы данных, которое в дальнейшем будет присутствовать в настройках. База данных включает в себя три таблицы
	\begin{enumerate}
		\item Coords - в таблице хранятся координатная и датчиковая информация
		
			\begin{verbatim}
		+------------+--------------+------+-----+---------+----------------+
		| Field      | Тип          | Null | Key | Default | Extra          |
		+------------+--------------+------+-----+---------+----------------+
		| id         | int(11)      | NO   | PRI | NULL    | auto_increment |
		| received   | datetime     | NO   |     | NULL    |                |
		| lat        | float        | YES  |     | NULL    |                |
		| lon        | float        | YES  |     | NULL    |                |
		| lat_b      | tinyblob     | YES  |     | NULL    |                |
		| lon_b      | tinyblob     | YES  |     | NULL    |                |
		| sensor     | tinyblob     | YES  |     | NULL    |                |
		| message_id | int(11)      | NO   | MUL | NULL    |                |
		| D1         | int(11)      | NO   |     | NULL    |                |
		| D2         | int(11)      | NO   |     | NULL    |                |
		| D3         | int(11)      | NO   |     | NULL    |                |
		| AL         | int(11)      | NO   |     | NULL    |                |
		| IG         | int(11)      | NO   |     | NULL    |                |
		| lat_s      | varchar(255) | YES  |     | NULL    |                |
		| lon_s      | varchar(255) | YES  |     | NULL    |                |
		+------------+--------------+------+-----+---------+----------------+
		\end{verbatim}
		\item Files - в таблице хранится информация о обработанных файлах.
		
			\begin{verbatim}
		+-----------+--------------+------+-----+---------+----------------+
		| Field     | Type         | Null | Key | Default | Extra          |
		+-----------+--------------+------+-----+---------+----------------+
		| id        | int(11)      | NO   | PRI | NULL    | auto_increment |
		| name      | varchar(255) | NO   | MUL | NULL    |                |
		| processed | int(11)      | NO   |     | NULL    |                |
		+-----------+--------------+------+-----+---------+----------------+
		\end{verbatim}
		\item Messages - в таблице хранится информация о сообщениях полученных путем анализа msg файлов

	\begin{verbatim}
+-----------+--------------+------+-----+---------+----------------+
| Field     | Type         | Null | Key | Default | Extra          |
+-----------+--------------+------+-----+---------+----------------+
| id        | int(11)      | NO   | PRI | NULL    | auto_increment |
| name      | varchar(255) | NO   | MUL | NULL    |                |
| processed | int(11)      | NO   |     | NULL    |                |
+-----------+--------------+------+-----+---------+----------------+

\end{verbatim}

	\end{enumerate} 


	
	\subsubsection { Настройка Samba}
	
	Настройка Samba заключается в создании директории где будут храниться загруженные с КПУС файлы и указании настроек этой директории. Настройки указываются в настроечном файле /etc/samba.conf.
	Для создания дирректории  необходимо набрать команду в консоли
	\begin{center}
		\fcolorbox{green}{lgreen}{\textit mkdir /samba/allacess}
	\end{center}
	Далее необходимо создать еще две поддиректории: рабочую и архивную. В рабочей будут храниться только что загруженные файлы, а в архивную будут складываться файлы после их обработки. 
	Для создания рабочей поддиректории наберите в консоли команду
	\begin{center}
		\fcolorbox{green}{lgreen}{mkdir /samba/allacess/exchange}
	\end{center}
	Для создания архивной директории наберите в консоли команду
	\begin{center}
		\fcolorbox{green}{lgreen}{mkdir /samba/allacess/storage}
	\end{center}
	После того как вы создали рабочие поддиректории необходимо  добавить следующую секцию в файл /etc/samba.conf. Откройте текстовый редактор и допишите в конец файла следующие строки:
	\begin {verbatim}
	[allaccess]
	path = /samba/allaccess
	browsable = yes
	writable = yes
	guest ok = yes
	read only = no
\end{verbatim}
где path - путь к созданной папке на сервере

	\subsection {Настройка специальных программных модулей}
	\subsubsection{Настройка ПМ <<Анализ>>}
	Настройка ПМ <<Анализ>> заключается в редактировании конфигурационного файла alarm.ini. Где в соответствующих секциях указываются параметры БД и папки где хранятся анализируемые сообщения 
	где
	\begin{enumerate}
		\item ftp\_dir - рабочая директория на Samba сервере, подмонтированная
		 к файловоой системе, указывается в секции [ftp]
		 \item db\_host - IP адрес узла, где расположена база данных, указыватеся в секции [bd]
		 \item db\_name - имя базы данных, указыватеся в секции [bd]
		 \item db\_login - логин пользователя БД, указыватеся в секции [bd]
		 \item db\_password - пароль пользователя БД, указыватеся в секции [bd]
	\end{enumerate}
	
	Пример настроек текущей системы
\begin{verbatim} 
[ftp]
ftp_dir=/home/server/storage/
[db]
db_host=192.168.20.199
db_login=alarm
db_name=mydb
db_password=alarmuser
\end{verbatim}
Для настройки автоматического запуска необходимо добавить в файл /etc/crontab
запись, которая обеспечивает периодический запуск скрипта alarm.sh
Пример

\begin{verbatim}
*/1 *   * * *   user    /home/user/ALARM_AUTO/alarm.sh
\end{verbatim}

	\subsubsection{Настройка на OC Ubuntu 16}
	Для настройки ПМ <<Загрузчик>> необходимо перенести файлы
	\begin{enumerate}
	\item alarm.pl
	\item alarm.ini
	\end{enumerate}
	в папку пользователя. Для пользователя user, его домашняя папка называется /home/user
	Далее необходимо произвести настройку демона cron, добавив в crontab файл, расположенный по адресу 
	/etc/crontab
	следующую запись
	\fcolorbox{green}{lgreen}{*/1 * * * * <Имя пользователя>  <Путь к папке программы>/alarm.pl}
	
	демон cron обеспечит автоматический запуск приложения в случае программного сбоя или перезагрузки системы.


	\subsubsection{Настройка доступа к файловому хранилищу}
	В случае настройки Linux необходимо примонтировать папку с сообщениями, которая затем настраивается в файле alarm.ini
	Для этого производим редактирования файла /etc/fstab добавляя автоматически монтируемую директорию
	//192.68.20.199/allaccess /home/server cifs username=guest, pass=guess
	\subsubsection{Настройка программного модуля выгрузки данных ПМ <<Загрузчик>>}
	Программный модуль выгрузки данных представляет из себя приложение написанное на языке perl. Приложение осуществляет выгрузку файловых сообщений из внешнего хранилища в локальное хранилище для дальнейшей обработки программным модулем обработки данных, кроме этого ПМ осуществляет резервирование выгруженных из хранилища данных.
	Располагается модуль на сервере по адресу
	/home/user/alarm.pl
	Файл настроек располагается в этой же папке 
	/home/user/alarm.ini
	запускается автоматически при перезапуске системы при помощи демона cron или в ручном режиме
	 \newline
    \fcolorbox{green}{lgreen}{/home/user/alarm.pl -f /home/user/alarm.ini}
    \newline
  

	Локальное хранилище необработанных данных представляет из себя сервер под управление ОС Ubuntu на котором установлео приложение Samba, позволяющее  получить доступ через протокол SMB/CIFS. Представлено двумя сетевыми папками
	\\<ip адрес сервера>\\allaccess\\storage
	в первую попадают сообщения из внешнего FTP хранилища и впоследствии удаляются после обработки
	\\<ip адрес сервера>\\allaccess\\exchange
	в эту папку попадают сообщения из внешнего FTP хранилища и здесь же и остаются
	с целью хранения
 
Настройка модуля синхронизации осуществляется путем правки конфигурационного файла где указываются следующие параметры:
\begin{verbatim}
[ftp]
ftp_login=< ЛОГИН ПОЛЬЗОВАТЕЛЯ КПУС ДЛЯ ДОСТУПА К FTP >
ftp_password=< ПАРОЛЬ ПОЛЬЗОВАТЕЛЯ КПУС >
ftp_dir=< ИМЯ ДИРЕКТОРИИ >
\end{verbatim}

где
[ftp] - название секции ini файла
ftp\_login - логин пользователя для доступа к FTP серверу КПУС
ftp\_password - пароль пользователя для доступа к FTP серверу КПУС
ftp\_dir - папка куда будет  скачиваться сообщения с FTP КПУС

Для каждого абонента КПУС необходимо запускать отдельную копию модуля синхронизации с соответствующими параметрами. Осуществляется это путем добавления записи в crontab файл, расположенный по адресу /etc/crontab

Пример записи:

	\fcolorbox{green}{lgreen}{*/1 * * * * user  /home/user/alarm.pl -f /home/user/alarm.ini}
	
	где:
	
user - имя пользователя от которого производится запуск скрипта


/home/user/alarm.pl - полный путь к скрипту синхронизации

/home/user/alarm.ini - конфигурационный файл для соответствующего абонента КПУС

\begin{enumerate}
\item Параметры запуска модуля синхронизации.
При запуске модуля синхронизации необходимо указывать ключ -f и его значение - путь к конфигурационному файлу.
\item Запуск модуля синхронизации
Модуль синхронизации можно запустить в ручном режиме и в автоматическом. В ручном режиме администратор выполняет запуск 
<имя скрипта> -f <имя конфигурационного файла>
В автоматическом режиме запуск производится самостоятельно системой при помощи демона cron

При добавлении нового абонента КПУС, его параметры прописываются в новом конфигурационном ini файле, после чего делается запись в crontab файл о запуске еще одного экземплера ПК <<Загрузчик>>  с указанием чтения настроек из нового конфигурационного файла
Пример:
\begin{verbatim}
*/1 * * * *     user    /home/user/alarm.pl -f /home/user/alarm.ini
#*/1 * * * *     user    /home/user/alarm2.pl -f /home/user/alarm2.ini
*/1 * * * *     user    /home/user/alarm3.pl -f /home/user/alarm3.ini

\end{verbatim}
В этом случае мы запускаем два экземпляра ПК <<Загрузчик>>, указав для каждого свой конфигурационный файл. Экземпляр загрузчика с конфигурационным файлом alarm2.ini в нашем случае отключен.
\end{enumerate}



	\subsubsection{Настройка системы отображения данных ПМ <<База>>}
	Для настройки ПМ <<База>> необходимо перейти в папку /home/gonec и произвести 
	настройку следующих конфигурационных файлов
	start\_puma.sh - файл отвечающий за автоматический запуск Web сервера Rails
	\begin{verbatim}
	#!/bin/bash
	cd /home/user/gonec
	export PATH="$HOME/.rbenv/bin:$PATH"
	echo $PATH
	eval "$(rbenv init -)"
	rails s -e production
	touch /home/user/started.pm
	\end{verbatim}

    config/database.yml - файл отвечающий за взаимодействие Web сервера с БД
\begin{verbatim}
	production:
	<<: *default
	database: mydb
	adapter: mysql2
	username: alarm
	password: alarmuser
\end{verbatim}	
 \begin{enumerate}
\item database - имя базы данных (мы задаем при создании БД  MySQL)
\item username - имя пользователя БД
\item password - пароль пользователя БД
  \end{enumerate}
	\section {Проверка работоспособности и дополнительные настройки ПК <<ТРЕВОГА>>}	
	\subsection{Проверка работоспособности подсистемы отображения данных ПМ <<База>>}
	Открыть в окне браузера 
	
	http://<ip адрес>:3000/messages
	
	Если не открывается то проверить доступность сервера при
	помощи встроенной в операционную систему утилиты ping.
	
	\fcolorbox{green}{lgreen}{ping <ip адрес>}
   
   
    В нашем случае следует открыть адрес
    
    \fcolorbox{green}{lgreen}{http://192.168.20.199/messages/}
    
    \subsection{Проверка работоспособности подсистемы обработки данных}
    В консоли набрать
    
    
    \fcolorbox{green}{lgreen}{ps -ax | grep alarm}
    
    
    Убедиться что процесс присутствует в списке процессов. Пример вывода утилиты ps, где отображена работа alarm.pl
    \begin{verbatim}
    user@uftp:~$ ps -ax | grep alarm
    12727 ?        Ss     0:00 /bin/sh -c    /home/user/alarm3.pl -f /home/user/alm3.ini
    \end{verbatim}
 \end{document}