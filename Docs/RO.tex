\documentclass[12pt]{article}[a4paper,14pt,russian]
\usepackage[russian]{babel}
\usepackage[utf8]{inputenc}
\usepackage{color}
\definecolor{lgreen}{rgb}{0.9,1,0.8}
\begin{document}
	\section {Назначение системы}
	Система ПК "Alarm" предназначена для выгрузки сообщений полученных от абонентских терминалов из ПК КПУС, анализа данных сообщений и представления полученной информации в удобной для пользователя форме посредством WEB браузера.
	
	\section{Состав системы}
	Система обработки данных представлена программным комплеком "Alarm" (Далее ПК "Alarm") состоит из следующих программных модулей.
	\begin{enumerate}
	\item  Специальные программные модули - программные модули разработанные программистами ОАО СС Гонец
	\item  Общесистемны программные модули  - программные модули разработанные сторонними организациями
	\end{enumerate}
    \subsection {Состав специальных программных модулей}
	\begin{enumerate}
	\item программный модуль отобржения данных "WEB" (далее ПМ "Web")
	\item программный модуль выгрузки и синхронизации данных "Synchro" (далее ПМ "Synchro")
	\item программный модуль анализа данных "Alarm" (далее ПМ "Alarm")
	\end{enumerate}
	\subsection {Состав общесистемных программных модулей}
    \begin{enumerate}
    \item программный модуль СУБД MySQL   свободная реляционная система управления базами данных
    \item программный модуль Samba - пакет программ, которые позволяют обращаться к сетевым дискам и принтерам на различных операционных системах по протоколу SMB/CIFS.
    
     \end{enumerate}

	\section{Описание специальных программных модулей}	
	\subsection{Программный модуль отображения данных ПМ "Web"}

	Программный модуль отображения данных представляет из себя приложение написанное
	на языке программирования Ruby с использование фреймворка Rails. Для своей работы требует сервер под управлением ОС Linux и наличие хранилища данных, реализованного в виде СУБД MySQL. Программный модуль осуществляет выборку и отображение из БД  сообщений полученных от абонентских терминалов, далее АТ, а также показывает координатную и датчиковую информацию содержащуюся в этом сообщение. В случае наличия датчиковой информации, которая трактуется нами как информация повышенного внимания, она маркируется красным цветом.
	
	\subsection{Программный модуль обработки данных ПМ "Alarm"}
	\subsubsection{Общее описание программного модуля}
	Программный модуль обработки данных представляет из себя приложение написанное на
	языке с++ с использование фреймворка QT. Программный модуль анализирует файловые сообщения, доступ к которомы получает посредством сетевой файловой системы из файлового хранилища данных. На основании анализа файловых сообщений ПМ производит заполнение БД, куда записываеются раскодированные и переформатированный данные полученные из файловых сообщений, после чего сообщение удаляется из хранилища. 
	\subsubsection{Минимальный состав технических средств}
	\begin{enumerate}
	\item	Для функционирования модуля необходимо иметь выделенный компьютер под управлением
	MS Windows 7 и выше либо OS Ubuntu 16
	\item Наличие сети ethernet
	\end{enumerate}
	\section {Настройка ПК "Alarm"}
	Настройка ПК "Alarm" включает в себя настройки специльных модулей и общесистемных модулей
	\subsection {Настройка общесистемных модулей}
	\subsubsection {Настройка СУБД MySQL}
	После установки СУБД MySQL необходимо создать пользователя для доступа к базе данных и непосредственно саму базу данных для хранения информации ПК "Alarm". База данных включает в себя три таблицы
	\begin{enumerate}
		\item Coords
		\item Files
		\item Messages
	\end{enumerate} 
	таблица
	\begin{verbatim}
	
	
	+------------+--------------+------+-----+---------+----------------+
	| Field      | Тип          | Null | Key | Default | Extra          |
	+------------+--------------+------+-----+---------+----------------+
	| id         | int(11)      | NO   | PRI | NULL    | auto_increment |
	| received   | datetime     | NO   |     | NULL    |                |
	| lat        | float        | YES  |     | NULL    |                |
	| lon        | float        | YES  |     | NULL    |                |
	| lat_b      | tinyblob     | YES  |     | NULL    |                |
	| lon_b      | tinyblob     | YES  |     | NULL    |                |
	| sensor     | tinyblob     | YES  |     | NULL    |                |
	| message_id | int(11)      | NO   | MUL | NULL    |                |
	| D1         | int(11)      | NO   |     | NULL    |                |
	| D2         | int(11)      | NO   |     | NULL    |                |
	| D3         | int(11)      | NO   |     | NULL    |                |
	| AL         | int(11)      | NO   |     | NULL    |                |
	| IG         | int(11)      | NO   |     | NULL    |                |
	| lat_s      | varchar(255) | YES  |     | NULL    |                |
	| lon_s      | varchar(255) | YES  |     | NULL    |                |
	+------------+--------------+------+-----+---------+----------------+
	\end{verbatim}
	\begin{verbatim}
	+-----------+--------------+------+-----+---------+----------------+
	| Field     | Type         | Null | Key | Default | Extra          |
	+-----------+--------------+------+-----+---------+----------------+
	| id        | int(11)      | NO   | PRI | NULL    | auto_increment |
	| name      | varchar(255) | NO   | MUL | NULL    |                |
	| processed | int(11)      | NO   |     | NULL    |                |
	+-----------+--------------+------+-----+---------+----------------+
	\end{verbatim}
	
	\begin{verbatim}
	+-----------+--------------+------+-----+---------+----------------+
	| Field     | Type         | Null | Key | Default | Extra          |
	+-----------+--------------+------+-----+---------+----------------+
	| id        | int(11)      | NO   | PRI | NULL    | auto_increment |
	| name      | varchar(255) | NO   | MUL | NULL    |                |
	| processed | int(11)      | NO   |     | NULL    |                |
	+-----------+--------------+------+-----+---------+----------------+
	
	\end{verbatim}
	\subsubsection {Настройка Samba}
	Настройка Samba заключается в создании дирректории где будут храниться загруженные с КПУС файлы и редактировании файла /etc/samba.conf. В этот файл необходимо добавить следующие строки.
	Для создания дирректории необходимо набрать команду
	\begin{center}

	\fcolorbox{green}{lgreen}{\textit mkdir /samba/allacess}

	\end{center}
	Далее необходимо создать еще две дирреткории: рабочую и архивную. В рабочей будут храниться только что загруженные файла, а в архивную будут складываться файлы после их обработки
	Для создания рабочей дирректории наберите в консоли
	\begin{center}
	\fcolorbox{green}{lgreen}{mkdir /samba/allacess/exchange}
	\end{center}
	Для создания архивной дирректории наберите в консоли
	\begin{center}
	\fcolorbox{green}{lgreen}{mkdir /samba/allacess/storage}
	\end{center}
	После того как вы создали рабочие дирректории необходимо добавить следующую секцию в файл /etc/samba.conf
	\begin {verbatim}
	[allaccess]
	path = /samba/allaccess
	browsable = yes
	writable = yes
	guest ok = yes
	read only = no
	\end{verbatim}
	где path - путь к созданной папке на сервере
	
	\subsubsection{Настройка ПМ "Alarm"}
	\subsubsection{Настройка на OC Ubuntu 16}
	Для настройки ПМ "Synchro" необходимо перенести файлы
	\begin{enumerate}
	\item alarm
	\item alarm.sh
	\item alarm.ini
	\end{enumerate}
	
	в папку пользователя пусть будет /home/user/ALARM/
	Далее добавить в cron файл запись 

	\fcolorbox{green}{lgreen}{*/1 * * * * <Имя пользователя>  <Путь к папке программы>/alarm.sh}
	cron обеспечит автоматический запуск приложения в случае программного сбоя или сбоя по питанию
	\subsection{Настройка общесистемных программных модулей}

	\subsubsection{Настройка доступа к файловому хранилищу}
	В случае настройки Linux необходимо примонтировать папку с сообщениями, котора затем настраивается в файле alarm.ini
	Для этого производим редактирования файла /etc/fstab добавляя автоматически монтируемую дирректорию
	//192.68.20.199/allaccess /home/server cifs username=guest, pass=guess
	\subsection{Программный модуль выгрузки данных "Synchro"}
	Программный модуль выгрузки данных представляет из себя приложение написанное на языке perl. Приложение осуществляет выгрузку файловых сообщений из внешнего хранилища в локальное хранилище для дальнейшей обработки программным модулем обработки данных, кроме этого ПМ осуществляет резервирование выгруженных из хранилища данных.
	Располагается модуль на сервере по адресу
	/home/user/alarm.pl
	Файл настроек располагается в этой же папке 
	/home/user/alarm.ini
	запускается автоматически при перезапуске системы при помощи демона cron или в ручном режиме
	 \newline
    \fcolorbox{green}{lgreen}{/home/user/alarm.pl -f /home/user/alarm.ini}
    \newline
   файл конфигурации alarm.ini
    ftp-login - логин для доступа к внешнему хранилищу
    ftp-password - пароль для доступа к внешнему хранилищу
	\subsection{Локальное хранилище необработанных данных "ALARM-STORAGE"}
	\section{Проверка работоспособности серверного ПО}
	Локальное хранилище необработанных данных представляет из себя сервер под управление ОС Ubuntu на котором установлео приложение Samba, позволяющее  получить доступ через протокол SMB/CIFS. Представлено двумя сетевыми папками
	\\<ip адрес сервера>\\allaccess\\storage
	в первую попадают сообщения из внешнего FTP хранилища и впоследствии удаляются после обработки
	\\<ip адрес сервера>\\allaccess\\exchange
	в эту папку попадают сообщения из внешнего FTP хранилища и здесь же и остаются
	с целью хранения
	\section{Настройка хранилища}
	Для автоматического подключения SAMBA необходимо прописать в файл
	/etc/fstab
	\begin{verbatim}
	//192.168.20.199/allaccess/     /home/server/   cifs username=guest,pass=guest  0       0
	\end{verbatim}
	где 192.168.20.199 - адресс локального хранилища
	/home/server - локальная папка куда монтируется папка /allaccess/ локального хранилища
	\subsection{Проверка работоспособности подсистемы отображения данных "Web"}
	Открыть в окне браузера http://<ip адресс>:3000/messages
	Если не открывается то проверить доступность сервера при
	помощи встроенной в операционную систему утилиты ping.
	
	\fcolorbox{green}{lgreen}{ping <ip адресс>}
    \subsection{Проверка работоспособности подсистемы обработки данных}
    В командной строке набрать 
    \fcolorbox{green}{lgreen}{ps -ax | grep alarm}
    \subsubsection{Добавление нового абонента для синхронизации}
    Модуль синхронизация реализован на скриптовом языке perl и состоит непосредственно
    из файла скрипта perl.pl и конфигурационного файла
    Убедиться в наличии работающего процесса alarm.
    
   \subsubsection{Настройка модуля синхронизации}
   Настройка модуля синхронизации осуществляется путем правки конфигурационного файла где указываются следующие параметры
   \begin{verbatim}
   [ftp]
   ftp_login=<ЛОГИН ПОЛЬЗОВАТЕЛЯ КПУС ДЛЯ ДОСТУПА К FTP>
   ftp_password=<ПАРОЛЬ ПОЛЬЗОВАТЕЛЯ КПУС>
   ftp_dir=<ИМЯ ДИРРЕКТОРИИ>
   \end{verbatim}
  
   где
    [ftp] - название секции ini файла
    ftp\_login - логин пользователя для доступа к FTP серверу КПУС
    ftp\_password - пароль пользователя для доступа к FTP серверу КПУС
    ftp\_dir - папка куда будет  скачиваться сообщения с FTP КПУС
    
    Для автоматического запуска модуля синхронизации для каждого абонента должна прописываться в 
    cron файле, где 
    /home/user/alarm - полный путь к скрипту синхронизации
    /home/user/alarm.ini - полный путь к конфигурационному файлу
    \subsubsection{Параметры запуска модуля синхронизации}
    При запуске модуля синхронизации необходимо указывать ключе -f и его значение - путь к конфигурационному файлу.
    \subsubsection{Запуск модуля синхронизации}
    Модуль синхронизации можно запустить в ручном режиме и в автоматическом. В ручном режиме администратор выполняет запуск 
    <имя скрипта> -f <имя конфигурационного файла>
	В автоматическом режиме запуск производится самостоятельно системой при помощи демона cron
    \begin{verbatim}
    */1 * * * *     user    /home/user/alarm.pl -f /home/user/alarm.ini
    \end{verbatim}
    \subsubsection{Проверка работоспособности подсистемы выгрузки данных}
	 
\end{document}